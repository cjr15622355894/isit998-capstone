%\documentclass[runningheads]{llncs}
\documentclass[12pt]{article}
\usepackage{amsfonts,amssymb}
\usepackage{plain}
\setcounter{tocdepth}{3}
\usepackage{color}
\usepackage{mdframed}
\usepackage{graphicx}
\usepackage{listings}
\usepackage{multicol}
%\usepackage{algorithmic,algorithm}
\usepackage{textcomp,booktabs}
\usepackage{graphicx,booktabs,multirow}
\usepackage{booktabs}
\newsavebox{\tablebox}
\usepackage{times}
\usepackage{hyperref}
\usepackage{ulem}

\topmargin -0.5cm \oddsidemargin 0cm \evensidemargin 0cm \textheight
23cm \textwidth 16cm




\renewcommand{\baselinestretch}{1.0}
 
\newcommand{\tb}{\textcolor{blue}}
\newcommand{\tr}{\textcolor{red}}

%-------------------------------------------------------------------------
\begin{document}

\title{Title of Project Proposal}


\author{
Name 1\\
Name 2\\
Name 3\\
Name 4\\
Name 5\\
}




\maketitle

\abstract{This project innovatively presents a city to tourists. Each Australian city has attractions including landmarks, buildings, monuments, etc. Currently tourists must browse the Internet or brochures to learn their stories. This is inconvenient during a tour, especially for foreigners. This calls for a convenient and instant information access mechanism. Considering everyone has a smart phone with camera, a mobile application can be developed with our image recognition research. By simply taking a picture of an attraction, tourists can instantly access its story, presented visually or verbally in their mother tongue. Functions such as photo-sharing and local advertisements can be added.

\tr{Put the paragraph of project description here. It should cover
\begin{itemize}
\item the background, 
\item the current solution and its disadvantages (that you focus on)
\item  and what this project will contribute
\end{itemize}
}

\newpage

\section{\tb{Introduction}}

\tr{
\begin{itemize}
\item The project proposal must have at least 10 pages including the title page but excluding references. 
\item You can insert some pics or tables to further explain what you are going to introduce. 
\item You must study how to cite references.  Note that part of contents of this proposal are allowed to be directly used in your final report.
\item You can use Google to understand how to write a project proposal
\item You are not allowed to change the format or font size of this Latex template.
\end{itemize}
}

\medskip


There are many ways of introducing the background of your project. Here, I introduce one template:

The introduction can be seen as the extension of the abstract.
\begin{itemize}
\item Introduce the background of problem you are going to mention.

\item Introduce the problem (why this is important)


\item Briefly introduce and summarize related solutions, and identify the gaps. Put the detailed introductions in the related work.


\item Introduce the aims in the subsection in details.

\item  Introduce related work  in the subsection in details.
\end{itemize}


\subsection{Aims}

\subsection{Related Work}


Briefly introduce each solution, advantage and disadvantage of each solution.

TIPS: You can introduce details about each solution if it is hard for you to write up to 10 pages.


\section{\tb{Methodology}}
The approach/methodology section describes the methods and materials that you intend to use within your project. It will detail how you will go about acquiring the information you need and explain how the project will be managed. If a particular theory or model is to be used, this should be stated. You could also include details such as timelines,locations, or the software to be used.





You can additionally mention others, but you must at least mention the following points in each subsection.

\subsection{Project Challenges}

Summarize all challenges or difficulties in this project and how you plan to solve them.


\subsection{Project Team}

Introduce your team (such as skills and How the tasks are scheduled)


\subsection{Plan and Time Line}

~~


~\\
\section{\tb{Outcomes}}

This section should include the expected impacts and benefits of the proposed project. It is important that these be specific and realistic.

Example: “Outcomes of this project will include:

\begin{itemize}
\item a prototype of an engrossing, original computer game of professional standard

\item An academic paper that solves XXX problems.

\item for team members, valuable insights, enhanced skills and an original work for portfolios.”
\end{itemize}

 
 
 
\tr{You can use any format of the reference as long as consistent. Pls search what is  Latex reference bib
}


\url{https://www.overleaf.com/learn/latex/Bibliography_management_with_bibtex}
 
 \begin{thebibliography}{9}
\bibitem{texbook}
Donald E. Knuth (1986) \emph{The \TeX{} Book}, Addison-Wesley Professional.

\bibitem{lamport94}
Leslie Lamport (1994) \emph{\LaTeX: a document preparation system}, Addison
Wesley, Massachusetts, 2nd ed.
\end{thebibliography}

\end{document}
